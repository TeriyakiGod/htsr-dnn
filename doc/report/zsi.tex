\documentclass{article}

\usepackage{graphicx}

\title{Text symbol classification using neural networks}
\author{Kacper Ochnik \and Paweł Frankowski}
\date{\today}

\begin{document}

% Create a title page
\begin{titlepage}
	\centering
	\includegraphics[width=0.3\textwidth]{Logo_PK_kolor_EN_PNG.png}\par\vspace{1cm}
	{\textsc{Koszalin University of Technology} \par}
	\vspace{1cm}
	{\Large \textsc{Applications of artificial intelligence project report}\par}
	\vspace{1.5cm}
	{\huge\bfseries Handwritten text symbol recognition with deep neural networks
	\par}
	\vspace{2cm}
	{\Large\itshape {Paweł Frankowski \space Kacper Ochnik}\par}
	\vfill
	supervised by\par
	Dr.~Adam Słowik

	\vfill

% Bottom of the page
	{\large \today\par}
\end{titlepage}

% Create a table of contentss
\tableofcontents
\newpage

\section{Introduction}
Handwritten text symbol recognition with deep neural networks.

\section{Our Goal}

The primary objective of our project is to develop a handwritten text symbol recognition system using deep neural networks. We aimed to create a model capable of accurately identifying and classifying handwritten digits ranging from 0 to 9 on a matrix of 28x28 pixels.

\subsection{Specific Objectives}

In pursuit of our overarching goal, we have identified the following specific objectives:

\begin{enumerate}
    \item \textbf{Project setup} - Set up the project environment and install the necessary libraries and packages.
    \item \textbf{Code implementation} - Write Python code to implement the deep neural network architecture. This includes developing modules for data preprocessing, model training, and evaluation.
    \item \textbf{Data Collection} - Gather a comprehensive dataset of handwritten digits (0 to 9) in a 28x28 pixel matrix format from MNIST.
	\item \textbf{Learning} - Train the model using the collected dataset. 
	\item \textbf{Optimization} - Improve the model's accuracy through optimization techniques. Accelerate computational efficiency for faster calculations.
    \item \textbf{Testing} - Create testing GUI for the trained model. Evaluate the model's performance metrics.
	\item \textbf{Documentation} - Write a comprehensive report documenting the project's objectives, methodology, and outcomes.
\end{enumerate}

\subsection{Expected Outcomes}

Upon successful completion of our project, we anticipate achieving the following outcomes:

\begin{itemize}
    \item Develop a robust deep neural network model capable of recognizing and classifying handwritten digits from 0 to 9.
    \item Train the model to achieve a acceptable level of accuracy.
\end{itemize}

\section{Decision boundary }
...

\section{Weights and biases - Pawel}
...

\section{Hidden layers}
...

\section{Activation functions - Pawel}
...

\section{Cost function}
...

\section{Gradient descent - Pawel}
...

\section{Cost landscape}
...

\section{Learning algorithm - naive approach - Pawel}
...

\section{Learning algorithm - calculus approach}
...

\section{Learning algorithm - digit recognition - Pawel}
...

\section{Chain rule - Pawel}
...

\section{Backpropagation}
...

\section{Testing the network}
...

\section{Conclusion}
This is the conclusion of the document.

\end{document}
